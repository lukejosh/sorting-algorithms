\documentclass{article}
\usepackage{amsmath}
\usepackage{algorithm}
\usepackage{algpseudocode}
\usepackage{algorithmicx}
\usepackage{enumitem}

\title{CAB302 - Assignment 1}
\author{Luke Josh}

\begin{document}
\maketitle
\tableofcontents
\pagebreak

\section{Introduction}
\subsection{Background}
Insertion sort is an algorithm that takes an input of an unsorted (or sorted) list of numbers, and rearranges
them into asscending order. It belongs to a group of algorithms known as sorting algorithms, but is by far from the best.
There are a number of more complex algorithms that can work much faster (especially as the size of the array increases),
however, it is extremely simple, easy to implement, and great as a tool to teach computer science students about algorithms.

The algorithm can be thought of as seperating the array into two subarrays, one that is sorted, and one that isn't. Initially,
the first element of the array is sorted - as there is nothing to compare it to. The remaining unsorted elements are then itteratively compared
to the sorted elements, and are placed into the index in which they belong. The process is often described as how a human would normally sort a hand of cards.
An implementation of the algorithm is included below:\linebreak

\subsection{Psuedocode}
\begin{algorithmic}
    \Function{InsertionSort}{$A[0..n-1$}
        \For{$i \leftarrow 1$ \bf{to} $n - 1$}
            \State{$v \leftarrow A[i]$}
            \State{$j \leftarrow i - 1$}
            \While{$j \geq 0$ \bf{and} $A[j] > v$}
                \State{$A[j + 1] \leftarrow A[j]$}
                \State{$j \leftarrow j - 1$}
            \EndWhile
            \State{$A[j + 1] \leftarrow v$}
        \EndFor
    \EndFunction
\end{algorithmic}

\subsection{Explanation}
In words, the algorithm can be expressed as follows:
\begin{enumerate}[itemsep=0mm]
    \item Take the nth element of the array, call this the subject element
    \item Compare the nth element with the n-1th element (the element to the left)
    \item If the element to the left is larger than the subject element, swap the two elements
    \item Else, if there is no item to the left, or the item to the left is less than the subject element, do nothing, and proceed along the array
    \item Repeat for all elements of the array
\end{enumerate}

\subsection{Example}
This can be observed in a simple example, take the array $A=[1, 3, 4, 2, 0]$ - the algorithm would observed the following steps:

\begin{enumerate}[itemsep=0mm]
    \item Take the zeroth element, $v=0$, there are no elements to the left, continue
    \item Take the first element, $v=1$, the element to the left is lower, continue
    \item ...
    \item Take the third element, $v=4$, the element to the left is greater, so call the third element, 4, the subject element
    \item Perform $A[j + 1] \leftarrow A[j]$, which becomes $A[3] \leftarrow A[2]$, thus $A=[1, 3, 4, 4, 0]$
    \item Repeat this operation along the list, until the item to the left is no longer less than the subject element: $A=[1, \mathbf{3}, 3, 4, 0]$
    \item Now, as the previous element is not less than the subject element, instead of making the element equal to the one before it ($A[j + 1] \leftarrow A[j]$), we set the element to be the subject element($A[j + 1] \leftarrow      v$), thus, $A=[1, 2, 3, 4, 0]$
    \item Repeat the process for the last element, which yields $A=[0, 1, 2, 3, 4]$
\end{enumerate}
\end{document}